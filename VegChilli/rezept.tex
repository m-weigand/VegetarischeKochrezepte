\section{Vegetarisches Chilli}
% Linke Seite: Rezept
Zutaten:
\begin{itemize}
    \item 2 Dosen Kidneybohnen
    \item ca. 200ml Bulgur
    \item 3 Dosen (ca. 600g) geschälte und gehackte Tomaten
    \item 2 Knoblauchzehen
    \item je 1 Tasse (gehackt/klein geschnitten):
        \begin{itemize}
            \item Zwiebeln
            \item Sellerie (oder Lauch)
            \item Möhren
            \item Paprika
        \end{itemize}
    \item 1/2 Zitrone
    \item 1 Tl gemahlener Kümmel
    \item 1 Tl Basilikum
    \item 1 Tl Chilli
    \item Salz und Pfeffer
    \item 3 El Tomatenmark
    \item 3 El trockener Rotwein
    \item Olivenöl
    \item (Beilage) 1 frisches Baguette
\end{itemize}

\noindent Zubereitung:

\noindent In einem großen Topf die Zwiebeln und das Chilli andünsten. Das
restliche Gemüse (außer den Tomaten) dazugeben. Alle vermischen und Salzen und
Pfeffern. Dann bei geringer Hitze ca. 15 Minuten dünsten, bis das Gemüse gar
ist (die Möhren testen!).

In einem zweiten Topf die Tomaten aufkochen und den Bulgur dazugeben. Ca. 20
Minuten ziehen lassen. Wenn der Bulgur danach noch nicht weich  ist, nochmal
etwas Wasser dazugeben und erneut aufkochen lassen.

Gemüse mit Rotwein ablöschen und Zitronensaft dazugeben. Tomatenmark und die
restlichen Gewürze untermischen.

Die Kidneybohnen und den Bulgur dazugeben. Alles vermischen und nochmal kurz
aufkochen. Mit Salz und Pfeffer abschmecken.

Mit frischem Baguette servieren.

% Recht Seite: Bild
%\newpage
\mbox{}
\vfill
\begin{center}
%    \includegraphics[width=\textwidth]{}
\end{center}
\vfill
\mbox{ }
\newpage
